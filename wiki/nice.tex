\chapter{Imperative programming}

Consider doing the laundry. If we have a common, house-hold washing machine at
our disposal --- the \emph{routine}, may look something like this:

\begin{enumerate}

\item split the laundry into whites and colours;

\item wash whites;

\item wash colours;

\item dry the laundry;

\item neatly pile up the laundry.

\end{enumerate}

In a conventional functional programming language, such as SML, this may end up
looking like this:

\begin{code}
(pile (dry (wash (split laundry))))
\end{code}

While comprehensible, this does not read very well. Most languages we're used
to, i.e. \wikipedia{Indo-european}{Indo-European} languages, read
left-to-right, top-to-bottom. Here, the flow of the program is expressed
right-to-left, and there is no notion of top-down descent at all. If anything,
there's a notion of bottom-up ascent, if we for instance, lay out the program
like this:

\begin{code}
(pile
  (dry
    (wash
      (split laundry))))
\end{code}

This reads like we're putting the cart before the horse. Indeed, some tasks ---
such as, doing the laundry, cooking a meal, writing a program, etc. --- are
inherently sequential. Functional programming languages can be
\explanation{syntactically}{A programming language has syntactic rules defining
the valid positioning of symbols. In this case, the syntactic rule is that the
name precedes the arguments in a function call.} unfit for writing such
programs well.

We will develop the notion of a \emph{well-written program} throughout these
lecture notes, but we begin with the following definition:

\begin{definition}

A well-written program is a well-read program.

\end{definition}

That is, a well-written program is comprehensible to another programmer.
Naturally, not any old programmer will do, but someone with relative knowledge
of the problem domain, and programming experience comparable, or superceeding
yours, should be able to comprehend and evaluate your programs. The intent is
to facilitate a peer-reviewed evolution of programs in general.

We were also somewhat \emph{declarative} when we wrote the functional program.
For instance, we've left out the fact that we should wash whites before we wash
colours. Actually, we didn't even mention that we split the laundry into whites
and colours, merely that we spit it, somehow\ldots\ We've left it up to the
\var{wash} and \var{split} functions to mention these details.

\begin{definition}

A declarative style of programming is a style where we tell the reader ``what''
we're doing, rather than ``how'' we're doing it.

\end{definition}

{\bf Assignment}

ASCII report.

The DIKU canteen is the only completely student-driven canteen on the
Copenhagen University campus. The canteen was established in 1971, and runs
till this day. As a student-driven canteen, even the accounting is done by
students. No biggie for Datalogy students, eh? We'll see.

It is possible to buy from the canteen in bulk. The usual order consists of
perhaps 30 bottles of soft drinks, and 30 bottles of beer. Although all soft
drinks have the same price, beer prices may vary depending on how enthusiastic
you are about beer. Indeed, DIKU alumni and research groups can be quite a
picky bunch when it comes to beer specialities. We can without loss of
generality constrain ourselves to beer.

Every time a bulk purchase is made, the items and quantities purchased are
noted in a journal, leaving it to accounting to handle later.

This is where you come in. We need to write a program that can process the
journal and produce a summary report in ASCII. The summary report should
process the entire journal and provide as output:

\begin{itemize}

\item the total number of times each type of beer was purchased;

\item the total revenue made from each type of beer;

\item the overall revenue from bulk purchases;

\item an overview of the development of the revenue over time.

\end{itemize}

\begin{verbatim}
Guld Tuborg 10
Grøn Tuborg 8
Leffe Brun 15
Leffe Blond 15
\end{verbatim}

The users will
sometimes however chose to buy abnormal quantities of various drinks.

- Canteen accounting. List of items with their prices, followed by event
  receipts.

- ASCII bar chart.

- Challenge of the week: ASCII line chart. Hint: don't drink and derive!

Tracing the attacker's IP.

- GUI in visual basic, LOL.

- Parse (csv-based) log files.

- Find comon behaviour:

- - Periodic access to the same resource.

- - Frequent access to different resources.
