\assignment{The UTF-8 report}

The DIKU cafeteria is a unique site on the Copenhagen University campus. It is
a completely student-driven cafeteria, established in 1974, and active till
this day. The cafeteria is a social gathering point of DIKU --- the place where
you dine, study and hang out.

The cafeteria relies on its users to participate in some of the daily
maintenance, while a group of dedicated students ensure its day-to-day
operations. As such, there is no paid staff, and the canteen is open 24/7/365,
always with student-friendly prices.

Since the cafeteria is completely student-driven, even accounting is done by
dedicated students. As they currently are busy grading your work, we ask you to
help out with some of the accounting.

The cafeteria sometimes sells beverages in bulk, e.g. for various DIKU events.
A typical order consists of perhaps 30 bottles of soft drinks, and 30 bottles
of beer.  Although all soft drinks are equally priced, beer prices vary
depending on type. We can therefore without loss of generality constrain
ourselves to beer.

Every time a bulk purchase is made, the items and quantities purchased are
recorded in a journal, leaving it to accounting to handle later. We need a
program that can process the journal and produce a summary report in ASCII to
be presented at the next cafeteria general assembly.

Given the entire journal, the summary report should contain:

\begin{itemize}

\item the total number of times each type of beer was purchased;

\item the total revenue made from each type of beer;

\item the overall revenue from bulk purchases;

\item an overview of the development of the revenue over time.

\end{itemize}

The program should as input, read the file \mono{./input}, and as output write
to the file \mono{./output}.

{\bf Input format}

The input format is as follows:

\begin{codebox}
\zi <number-of-products> <number of purchases>
\zi <product-name> <product-price>
\zi <product-name> <product-price>
\zi ...
\zi <product-index> <quantity>
\zi <product-index> <quantity>
\zi ...
\end{codebox}

The \mono{<number-of-products>} and \mono{<number-of-purchases>} are provided
for your convenience. The former is an integer in the range [0;31], while the
latter is an integer in the range [0;1023].

Next comes a list of products, and a list of journal entries.

A product is represented by its name and price, separated by a space. The
product name is a string of characters, at most 42 characters in length. The
characters may be digits, letters in the Danish alphabet, or the space
character, in UTF-8 encoding. The product price is represented in whole
monetary units, and is in the range [0;127].

Every entry in the journal represents a purchase of a particular product. If
more than one product is actually purchased, this is split across multiple
entries in the journal. An entry contains the index of the purchased product,
the quantity purchased, and the time of the purchase, all separated by a space.
The product index is the index of the product from the top of the list of
products at the beginning of the file, starting at 0.  The quantity is an
integer in the range [0;1023]. The time is the number of seconds passed since
the last general assembly (held annually).

Here's the contents of a sample input file:

\begin{codebox}
\li 4 5
\li Guld Tuborg 10
\li Grøn Tuborg 8
\li Ale 10
\li Leffe 15
\li 0 30 86400
\li 0 30 86400
\li 3 30 172800
\li 2 180 259200
\li 1 1 432000
\end{codebox}

That is, the first purchase is 60 bottles of Guld Tuborg, 1 day after the
general assembly, followed by 30 bottles of Leffe 2 days after the general
assembly, followed by 180 bottles of Ale, and 1 bottle of Grøn Tuborg.

{\bf Output format}

Here's the expected output of the above sample input:

\begin{code}
+-------------+-------+----------+--------+
|   Product   | Price | Quantity | Income |
+-------------+-------+----------+--------+
| Guld Tuborg |    10 |       23 |     23 |
| Grøn Tuborg |     8 |       23 |     23 |
| Ale         |    10 |       23 |     23 |
| Leffe       |    15 |       23 |     23 |
+-------------+-------+----------+--------+
Most profitable: Ale
Sales over time:
+----+
|    |
|    |
| $$ |
\end{code}

- Challenge of the week: ASCII line chart. Hint: don't drink and derive!

Tracing the attacker's IP.

- GUI in visual basic, LOL.

- Parse (csv-based) log files.

- Find comon behaviour:

- - Periodic access to the same resource.

- - Frequent access to different resources.

Extra question: can integer overflow occur?

Next assignment: shuffle, and change the input format to not include the number
of purchases (or the number of products?).

Only ASCII letters, because Go is rather unfortunate with string lengths
otherwise.

Assume purchases to be in chronological order.
