{\bf Assignment}

ASCII report.

The DIKU canteen is the only completely student-driven canteen on the
Copenhagen University campus. The canteen was established in 1971, and runs
till this day. As a student-driven canteen, even the accounting is done by
students. No biggie for Datalogy students, eh? We'll see.

It is possible to buy from the canteen in bulk. The usual order consists of
perhaps 30 bottles of soft drinks, and 30 bottles of beer. Although all soft
drinks have the same price, beer prices may vary depending on how enthusiastic
you are about beer. Indeed, DIKU alumni and research groups can be quite a
picky bunch when it comes to beer specialities. We can without loss of
generality constrain ourselves to beer.

Every time a bulk purchase is made, the items and quantities purchased are
noted in a journal, leaving it to accounting to handle later.

This is where you come in. We need to write a program that can process the
journal and produce a summary report in ASCII. The summary report should
process the entire journal and provide as output:

\begin{itemize}

\item the total number of times each type of beer was purchased;

\item the total revenue made from each type of beer;

\item the overall revenue from bulk purchases;

\item an overview of the development of the revenue over time.

\end{itemize}

\begin{verbatim}
Guld Tuborg 10
Grøn Tuborg 8
Leffe Brun 15
Leffe Blond 15
\end{verbatim}

The users will
sometimes however chose to buy abnormal quantities of various drinks.

- Canteen accounting. List of items with their prices, followed by event
  receipts.

- ASCII bar chart.

- Challenge of the week: ASCII line chart. Hint: don't drink and derive!

Tracing the attacker's IP.

- GUI in visual basic, LOL.

- Parse (csv-based) log files.

- Find comon behaviour:

- - Periodic access to the same resource.

- - Frequent access to different resources.
