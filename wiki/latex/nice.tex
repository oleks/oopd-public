\chapter{Imperative, Procedural and Structured Programming}

Consider doing the laundry. If we have a common, house-hold washing machine at
our disposal, the \emph{procedure} may look something like this:

\begin{code}
split the laundry into whites and colours;
wash whites;
wash colours;
dry the laundry;
neatly stack up the laundry.
\end{code}

In a conventional functional programming language, such as SML, this may end up
looking like this:

\begin{code}
(stack (dry (wash (split laundry))))
\end{code}

While comprehensible, this does not read very well. Most languages we're used
to, read left-to-right, top-to-bottom. Here, the flow of the program is
expressed right-to-left, and there is no notion of top-down descent at all. If
anything, there's a notion of bottom-up ascent, if we for instance, lay out the
program like this:

\begin{codebox}
\li (stack
\zi   (dry
\zi     (wash
\zi       (split laundry))))
\end{codebox}

So we ``stack the dried, washed and split laundry''? This reads like we're
putting the cart before the horse. Some tasks, such as doing the laundry,
cooking a meal, writing a program, are inherently sequential, i.e. they are
done by performing a sequence of more basic tasks.  Functional programming
languages can be \explain{syntactically}{A programming language has syntactic
rules defining the valid positioning of symbols. In this case, the syntactic
rule is that the name precedes the arguments in a function call.} unfit for
writing programs to perform such tasks.

We will develop the notion of a \emph{well-written program} throughout these
lecture notes, but we begin with the following definition:

\begin{definition}

A well-written program is a well-read program.

\end{definition}

That is, a well-written program is comprehensible to another programmer.
Naturally, not any old programmer will do, but someone with relative knowledge
of the problem domain, and programming experience comparable, or superceeding
yours, should be able to comprehend and evaluate your programs. The intent is
to facilitate the development of peer-reviewed programs.

We were also somewhat \emph{declarative} when we wrote the functional program.
For instance, we've left out the fact that we should wash whites before we wash
colours. Actually, we didn't even mention that we split the laundry into whites
and colours; we've left it up to the \var{wash} and \var{split} functions to
discuss such ``details''.

\begin{definition}

A declarative style of programming is a style where we tell the reader ``what''
we're doing, rather than ``how'' we're doing it.

\end{definition}

An alternative functional program could be:

\begin{code}
let
  (dirtyWhites, dirtyColours) = split dirtyLaundry
  cleanWetWhites = wash dirtyWhites
  cleanWetColours = wash dirtyColours
  cleanLaundry = dry (cleanWetWhites, cleanWetColours)
in
  stack cleanLaundry
\end{code}

We'll refer to lines 1--7 as a ``\mono{let}-block'', lines 2--4 as a
``\mono{let}-definitions'', and line 7 as a ``\mono{let}-expression''.

In functional programming, the \mono{let}-syntax can be used to reduce the
complexity of an expression by giving intuitive names to some of its
constituents. This lets the reader read a simpler expression, and delve into
the details only if necessary. So long as the important details are retained in
the \mono{let}-expression, this is a useful technique for making your programs
more readable.

However, in this particular case, the expression reduces to the rather mundane
``\code{stack cleanLaundry}''. This seems overly declarative. We're completely
disregarding the important aspects of ``doing the laundry''. We're so eager not
to tell the reader ``how'' we're doing it, that we forget to tell the reader
``what'' we're doing.

The \mono{let}-syntax does however provide for a segway to another programming
paradigm, more suitable for writing procedures like this.

\begin{definition}

A programming paradigm is a particular way of writing a certain class of
programs well.

\end{definition}

In particular, the \mono{let}-syntax provides more than a simple mechanism for
naming subexpressions. Notice, how we could use the name \var{dirtyWhites} when
we defined \var{cleanWetWhites}, or how we could use \var{cleanWetWhites} when
we defined \var{cleanLaundry}.

It is as if some sort of ``state'' evolves from one defintion to the next, in
the sense that a name defined further up can be used in definitions further
down. If we represent this ``state'' as the set of names available in the
context of an expression, we can specify how the state develops throughout the
\mono{let}-block: 

\begin{codebox}
\li let
\li   {dirtyLaundry}
\li   {dirtyLaundry, dirtyWhites, dirtyColours}
\li   {dirtyLaundry, dirtyWhites, dirtyColours,
\zi     cleanWetWhites}
\li   {dirtyLaundry, dirtyWhites, dirtyColours,
\zi     cleanWetWhites, cleanWetColours}
\li in
\li   {dirtyLaundry, dirtyWhites, dirtyColours,
\zi     cleanWetWhites, cleanWetColours, cleanLaundry}
\end{codebox}

It seems overly excessive to have the state at the \mono{let}-expression (line
7) contain all these ``intermediate'' names, such as \var{dirtyWhites},
\var{cleanWetColours}, etc. All we really want in the \mono{let}-expression is
the \var{cleanLaundry}. What's more, it does not conceptually make sense to be
able to use the dirty, or wet laundry once the laundry has been washed and
dried, right? Indeed, we'd like for the state develop in a fashion similar to
this:

\begin{code}
{dirtyLaundry}
{dirtyWhites, dirtyColours}
{cleanWetWhites, cleanWetColours}
{cleanDryLaundry}
\end{code}

While we could attempt to attain this using nested \mono{let}-blocks or
\explain{auxiliary functions}{Functions whos sole purpose is to make other
functions more readable.}, the code will not be as easy as 1-2-3, i.e.  as
simple as our initial specification of the procedure.

The thing that is getting in the way here is that in the functional paradigm
we're dealing with \emph{immutable} values. That is, values that can be created
and destroyed, but not modified. Indeed, if the pile of laundry was a
\emph{mutable} value, we could simply \emph{sequentially} modify that value
from a dirty pile to a clean stack. After-all, this resembles what we do with
the laundry in real life.

\begin{definition}

A mutable value, or a ``variable'', is a named entity that can refer to
different concrete values throughout the lifetime of a program.

\end{definition}

Let us thus focus in on a single statement:

\begin{code}
dirtyWhites, dirtyColours = split dirtyLaundry
\end{code}

If we let \var{dirtyLaundry} initially be a basket of mixed dirty laundry,
we'de like for the \emph{effect} of this statement to be that \var{dirtyWhites}
becomes a pile of dirty whites, and \var{dirtyColours} becomes a pile of dirty
colours, while \var{dirtyLaundry} becomes an \emph{empty} basket. That is, we'd
like for the laundry to be \emph{moved} from the basket to the respective
piles.

At this point, it no longer makes sense to represent ``state'' as the mere set
of names available in the context of an expression. It now also matters which
concrete value a name refers to in the context of the expression.  Indeed,
before the laundry is split, \var{dirtyLaundry} is a basket full of dirty
clothes; after it, it should be an empty basket. If we represent baskets and
piles as good old lists, we'd like for the state to evolve from

\begin{codebox}
\li { dirtyLaundry = [A, B, C, D, ...] }
\end{codebox}

to

\begin{codebox}
\li {
\zi   dirtyLaundry = [],
\zi   dirtyWhites = [A, C, ...],
\zi   dirtyColours = [B, D, ...]
\zi }
\end{codebox}

in the course of the \var{split} procedure. Let us now specify this procedure:

\begin{codebox}
\li instanciate a whites pile;
\li instanciate a colours pile;
\zi  
\li if the basket is empty, end the procedure;
\zi  
\li pick up a piece of laundry from the basket;
\li if the piece is white, throw it in the whites pile;
\li otherwise, throw it in the colours pile;
\zi  
\li go to 3.
\end{codebox}

In this specification we've used the word ``instanciate''. In the context of
doing the laundry, this simply means finding a large enough empty spot on the
floor. Once we've found such a spot, we'd like to keep that spot in mind, so
that we know where to throw pieces of laundry after we pick them up from the
basket and inspect them. We hence retain a \emph{reference} to this spot.
After the steps 1--2 our state looks like this:

\begin{codebox}
\li {
\zi   dirtyLaundry = [A, B, C, D, ...],
\zi   dirtyWhites = [],
\zi   dirtyColours = []
\zi }
\end{codebox}

Notice, how \var{dirtyLaundry} is also merely a reference.

Steps 3--7 define an \emph{iterative} procedure, in the sense that the last
thing that happens in the procedure is a jump back its beginning.

Indeed, in SML, we could define this procedure as follows:

\begin{code}
split [] dirtyWhites dirtyColours =
  (dirtyWhites, dirtyColours)
split (piece :: dirtyLaundry) dirtyWhites dirtyColours =
  if isWhite piece
  then
    split dirtyLaundry (piece :: dirtyWhites) dirtyColours
  else
    split dirtyLaundry dirtyWhites (piece :: dirtyColours)
\end{code}

Although most imperative programming languages have retained the \var{goto}
statement, it is considered harmful. Instead, programmers are encouraged to use
various built-in \emph{control structures}, i.e. built-in higher-order
functions with native language support.

Wrt. defining iterative procedures, imperative languages often come with a
series of \emph{loop} control structures.

\begin{definition}

A loop control structure repeats a sequence of steps.

\end{definition}

One such control structure is the \emph{while} loop.

\begin{definition}

A while loop control structure repeats a sequence of steps as long as a boolean
condition is met.

\end{definition}

Lines 3--7 of the split procedure can be rewritten using a while loop in the
following fashion:

\begin{code}
while (basket is not empty)
{
  pick up a piece of laundry from the basket;
  if the piece is white, throw it in the whites pile;
  otherwise, throw it in the colours pile;
}
\end{code}
