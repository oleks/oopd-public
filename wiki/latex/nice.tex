\chapter{Imperative and procedural programming}

Consider doing the laundry. If we have a common, house-hold washing machine at
our disposal, the \emph{procedure} may look something like this:

\begin{enumerate}

\item split the laundry into whites and colours;

\item wash whites;

\item wash colours;

\item dry the laundry;

\item neatly stack up the laundry.

\end{enumerate}

In a conventional functional programming language, such as SML, this may end up
looking like this:

\begin{code}
(stack (dry (wash (split laundry))))
\end{code}

While comprehensible, this does not read very well. Most languages we're used
to, i.e. \wikipedia{Indo-european}{Indo-European languages}, read
left-to-right, top-to-bottom. Here, the flow of the program is expressed
right-to-left, and there is no notion of top-down descent at all. If anything,
there's a notion of bottom-up ascent, if we for instance, lay out the program
like this:

\begin{code}
(stack
  (dry
    (wash
      (split laundry))))
\end{code}

This reads like we're putting the cart before the horse. Indeed, some tasks ---
such as, doing the laundry, cooking a meal, writing a program, etc. --- are
inherently sequential, i.e. they are done by performing a sequence of steps.
Functional programming languages can be \explain{syntactically}{A programming
language has syntactic rules defining the valid positioning of symbols. In this
case, the syntactic rule is that the name precedes the arguments in a function
call.} unfit for writing such programs well.

We will develop the notion of a \emph{well-written program} throughout these
lecture notes, but we begin with the following definition:

\begin{definition}

A well-written program is a well-read program.

\end{definition}

That is, a well-written program is comprehensible to another programmer.
Naturally, not any old programmer will do, but someone with relative knowledge
of the problem domain, and programming experience comparable, or superceeding
yours, should be able to comprehend and evaluate your programs. The intent is
to facilitate the development of peer-reviewed programs.

We were also somewhat \emph{declarative} when we wrote the functional program.
For instance, we've left out the fact that we should wash whites before we wash
colours. Actually, we didn't even mention that we split the laundry into whites
and colours; we've left it up to the \var{wash} and \var{split} functions to
discuss such ``details''.

\begin{definition}

A declarative style of programming is a style where we tell the reader ``what''
we're doing, rather than ``how'' we're doing it.

\end{definition}

An alternative functional program could be:

\begin{code}
let
  (dirtyWhites, dirtyColours) = split dirtyLaundry
  cleanWetWhites = wash dirtyWhites
  cleanWetColours = wash dirtyColours
  cleanLaundry = dry (cleanWetWhites, cleanWetColours)
in
  stack cleanLaundry
\end{code}

We'll refer to lines 1--7 as a ``\mono{let}-block'', lines 2--4 as a
``\mono{let}-definitions'', and line 7 as a ``\mono{let}-expression''.

In functional programming, the \mono{let}-syntax can be used to reduce the
complexity of an expression by giving intuitive names to some of its
constituents. This lets the reader read a simpler expression, and delve into
the details only if necessary. So long as the important details are retained in
the \mono{let}-expression, this is a useful technique for making your programs
more readable.

However, in this particular case, the expression reduces to \code{stack
cleanLaundry}. This seems overly declarative. We're completely disregarding the
important aspects of ``doing the laundry''. We're so eager not to tell the
reader ``how'' we're doing it, that we forget to tell the reader ``what'' we're
doing.

The \mono{let}-syntax does however provide a pathway to introducing a different
programming paradigm, more suitable for writing \emph{procedures}.

\begin{definition}

A programming paradigm is a particular way of writing a certain class of
programs well.

\end{definition}

In particular,
the \mono{let}-syntax provides more than a simple mechanism for naming
subexpressions. Notice, how we could use the name \var{dirtyWhites} when we
defined \var{cleanWetWhites}, or how we could use \var{cleanWetWhites} when we
defined \var{cleanLaundry}.

It is as if some sort of ``state'' evolves from one defintion to the next, in
the sense that a name defined further up can be used in definitions further
down. If we represent this ``state'' as the set of names available in the
context of an expression, we can specify how the state develops throughout the
\mono{let}-block: 

\begin{code}
let
  {dirtyLaundry}
  {dirtyLaundry, dirtyWhites, dirtyColours}
  {dirtyLaundry, dirtyWhites, dirtyColours,
    cleanWetWhites}
  {dirtyLaundry, dirtyWhites, dirtyColours,
    cleanWetWhites, cleanWetColours}
in
  {dirtyLaundry, dirtyWhites, dirtyColours,
    cleanWetWhites, cleanWetColours, cleanLaundry}
\end{code}

It seems overly excessive to have the state at the \mono{let}-expression (lines
9--10) contain all these ``intermediate'' names, such as \var{dirtyWhites},
\var{cleanWetColours}, etc. All we really want in the \mono{let}-expression is
the \var{cleanLaundry}. What's more, it does not conceptually make sense to be
able to use the dirty, or wet laundry once the laundry has been washed and
dried, right? Indeed, we'd like for the state develop in a fashion similar to
this:

\begin{code}
{dirtyLaundry}
{dirtyWhites, dirtyColours}
{cleanWetWhites, cleanWetColours}
{cleanDryLaundry}
\end{code}

While we could attempt to attain this using nested \mono{let}-blocks or
\explain{auxiliary functions}{Functions whos sole purpose is to make other
functions more readable.}, the code will not be as easy as 1-2-3, i.e.  as
simple as our initial specification of the procedure.

The thing that is getting in the way here is that in the functional paradigm
we're dealing with \emph{immutable} values. That is, values that can be created
and destroyed, but not modified. Indeed, if the pile of laundry was a
\emph{mutable} value, we could simply step-wise modify that value from a dirty
pile to a clean stack. After-all, this resembles what we do with the laundry in
real life.

\begin{definition}

A mutable value, or a ``variable'', is a named entity that can refer to
different concrete values throughout the lifetime of a program.

\end{definition}

Let us thus focus in on a single statement:

\begin{code}
dirtyWhites, dirtyColours = split dirtyLaundry
\end{code}

If we let \var{dirtyLaundry} initially be a basket of mixed dirty laundry,
we'de like for the \emph{effect} of this statement to be that \var{dirtyWhites}
becomes a pile of dirty whites, and \var{dirtyColours} becomes a pile of dirty
colours, while \var{dirtyLaundry} becomes an \emph{empty} basket. That is, we'd
like for the laundry to be \emph{moved} from the basket to their respective
(recently created) piles.

At this point, it no longer makes sense to represent ``state'' as the mere set
of names available in the context of an expression. It now also matters which
\emph{concrete value} the name refers to in the context of an expression.
Indeed, before the above statement, \var{dirtyLaundry} is a basket full of
dirty clothes; after it, it is an empty basket.

In particular, if we represent baskets and piles as good old lists, we'd like
for the state to evolve from

\begin{code}
{ dirtyLaundry = [A,B,C,D,...] }
\end{code}

to

\begin{code}
{ dirtyLaundry = [],
  dirtyWhites = [A,C,...], dirtyColours = [B,D,...] }
\end{code}

Let us now consider the \var{split} procedure: 

\begin{enumerate}

\item allocate space for the whites pile;

\item allocate space for the colours pile;

\item if the basket is empty, end the procedure;

\item pick up a piece of dirty laundry;

\item if the piece is white, throw it in the whites pile;

\item else, throw it in the colours pile;

\item go to 3.

\end{enumerate}
