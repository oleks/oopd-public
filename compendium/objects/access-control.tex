\section{Access control}\label{section:objects:encapsulation}

% Prerequisites: array, dynamic array, inheritance, objects as data structures,
% object invariants, packages. 

Since objects are first and foremost data structures, they have to maintain
certain invariants in order to stay safe for others to use. On the one hand
there is the problem of designing an object's methods in such a way that
invariants are not broken, and on the other hand there is the problem of
ensuring that no adversary can break these invariants through access to the
underlying data representation. In this case, the adversary is the humble
programmer.

The process of \emph{modifying} who can access certain fields and methods of an
object, is called \key{access control}, and is done using \key{access
modifiers}. By far the most languages come with default access settings,
therefore it is important to understand that access modifiers indeed
\emph{modify} an access setting rather than set it. [TODO: provide a list of
languages and access modifiers here].

When is it a good idea to make the fields of an object private? One convention
is to keep \emph{all} the fields private. As a matter of fact, a language like
Smalltalk enforces this in the language semantics, i.e. fields cannot be
non-private. If any access to the data in the fields is necessary, appropriate
methods should provide access to that data in such a way that an adversary
cannot corrupt the object invariants.

Although this is almost surely invariant-preserving (the programmer has to be
careful about what these ``appropriate methods'' return), this can be thought
of as overly protective. Sometimes, exposing internals is a good thing. In
particular, if you're designing a data structure with another data structure
underneath, and the underlying data structure already provides certain methods
that you'd like for your new data structure to provide. Said otherwise, you
would like to provide an interface akin to the one of the underlying data
structure, and the implementation should merely delegate the operations to the
underlying data structure.

For instance, if we consider our \code{DynamicArray}, it can, and certainly
should, have $O(1)$ random access capabilities. After-all, the underlying data
structure, an array, has them.

, since the underlying data structure, an
array, has this property. This capability would allow for a general-purpose
swap method to be defined, which in turn would allow for a general-purpose sort
operation to be defined, that sorts the elements of a \code{DynamicArray}
without any extra space requirements.




Although safe, in the sense that the data structure invariants will be
maintained (as long as the programmer is clever about what these accessible
methods return) this is also overly protective. Sometimes, exposing internals
is a good thing.



Delegation of such methods from your data structure to the underlying data
structure is not always trivial.


For
instance, consider building a series different of containers with random access.


if we can implement a general-purpose swap function, we can implement
a general-purpose sorting function that sorts the elements in the container . If we have some sort of a container, and merely get and set functions, such a swap function is unattainable unless 

 which is useful if we
want to implement a general-purpose sort function.

However, this does not mean that they should be exposed in such a manner that
makes it possible for an adversary to break the data structure invariants.

The drastic approach of making all fields private has another benefit however.
It is highly maintainable in the sense that if there ever was a need to change
the underlying implementation, all of the code that is interacting with objects
of this type would still work as the there are no changes imposed on the object
interface by a concrete implementation change.

\subsection*{Summary}

\begin{itemize}

\item\item

\item For clarity, always modify the accessibility of classes components.

\item Open access in so far it makes sense to the corresponding interface depth.

\item Limit access such that data structure invariants cannot be broken by an
adversary.

\end{itemize}
