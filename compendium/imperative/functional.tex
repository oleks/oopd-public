What you've been practising thus far, is mostly the \key{functional}
programming paradigm.

\begin{definition}

Functional programming treats computation as application of mathematical
functions to \key{immutable} data.

\end{definition}

Programming using the functional paradigm is done by \key{defining}
mathematical functions. The execution of a program is then the application of
some particular \key{entry function} to some, if any, program arguments.
Functions themselves are defined in terms of function application and branching
of execution depending on the outcomes of boolean predicates.

A mathematical function defines a relation between a set of possible inputs,
its \key{domain}, and a set of possible outputs, its \key{codomain}. For a
function to be well-defined, every value in its domain of must have a
\emph{unique} \key{representation}, as must every value in its codomain.  A
function can then \key{consume} a representation of a value in its domain, and
\key{produce} a representation of a corresponding value in its codomain. We say
that a function takes a value as \key{input}, and return a value as
\key{output}.

% Domain and codomain representations do not have to unique wrt. to one
% another.

Due to this pattern of consumption and production, we say that all data is
immutable; i.e., there is no notion of one representation being mutated
(modified) into another. Mutation; however, is useful in practise. It is not
unlikely that the output of a function has a representation very close to the
representation of its input. In such cases, mutation can save us a great deal
of trouble when preparing a value to output.

Functional programming languages usually hide this aspect from the programmer,
letting the language itself worry about such ``premature'' matters. We say that
in a functional paradigm we're less concerned with the flow of data throughout
our program, and more concerned with the mappings that our functions define.
